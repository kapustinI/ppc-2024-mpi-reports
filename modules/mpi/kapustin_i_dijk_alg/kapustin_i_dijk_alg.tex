\documentclass[a4paper,14pt]{extarticle}
\usepackage[utf8]{inputenc}
\usepackage[T2A]{fontenc}
\usepackage[russian]{babel}
\usepackage{geometry}
\usepackage{hyperref}
\usepackage{listings}
\usepackage{xcolor}
\geometry{a4paper, top=30mm, left=35mm, right=20mm, bottom=25mm}

\begin{document}

\begin{titlepage}
    \begin{center}
        \vspace*{-2cm} % Смещение вверх
        {\large \textbf{«Национальный исследовательский Нижегородский государственный университет»}}
        
        \vspace{4cm} % Расстояние между названием университета и заголовком работы
        {\LARGE \textbf{Реализация алгоритма Дейкстры с использованием технологии MPI}}
        
        \vspace{3cm} % Расстояние между заголовком работы и автором
        \textbf{Автор работы:}\\
        \vspace{0.5cm}
        \textbf{Капустин Иван Александрович}\\
        Группа 3822Б1ПР1
        
        \vspace{1.5cm}
        \textbf{Преподаватель:}\\
        \vspace{0.5cm}
        \textbf{Сысоев Александр Владимирович}\\
        Кандидат технических наук, доцент
        
        \vfill % Заполнение оставшегося пространства
        Нижний Новгород, \the\year{} г.
    \end{center}
\end{titlepage}

\section*{Введение}

Алгоритм Дейкстры является одним из самых известных и широко используемых методов поиска кратчайших путей в графе. Основная идея алгоритма заключается в последовательном нахождении минимального пути от заданной стартовой вершины до всех остальных вершин графа. Алгоритм работает только с графами, в которых все рёбра имеют неотрицательные веса, что обеспечивает корректность вычислений.
Основные цели работы:
\begin{itemize}
    \item Разработка и реализация последовательной версии алгоритма.
    \item Разработка и реализация параллельной версии алгоритма.
    \item Сравнение результатов скорости работы последовательной и параллельной версий алгоритмов.
\end{itemize}
\section*{Описание последовательной версии алгоритма}

\subsection*{Конвертация матрицы смежности в формат CRS}

Функция \texttt{CRSconvert} отвечает за преобразование входной матрицы смежности в формат разреженного хранения CRS (Compressed Row Storage). Этот формат позволяет эффективно хранить и обрабатывать графы с большим количеством вершин и рёбер. В рамках данной функции:
\begin{itemize}
    \item Вектор \texttt{values} хранит ненулевые значения рёбер.
    \item Вектор \texttt{columns} содержит индексы вершин, соответствующих этим рёбрам.
    \item Вектор \texttt{row\_ptr} указывает начало и конец диапазона ненулевых значений для каждой строки.
\end{itemize}
Итерация по строкам матрицы происходит с использованием указателей, что обеспечивает высокую производительность.

\subsection*{Предварительная обработка данных}

Метод \texttt{pre\_processing} выполняет следующие действия:
\begin{enumerate}
    \item Извлекает матрицу смежности и параметры графа из входных данных.
    \item Вызывает функцию \texttt{CRSconvert} для преобразования данных в формат CRS.
    \item Инициализирует массив результатов \texttt{res\_} с начальными значениями: все расстояния установлены как бесконечность (\texttt{INF}), кроме начальной вершины, расстояние до которой равно 0.
\end{enumerate}

\subsection*{Основной алгоритм}

Метод \texttt{run} реализует сам алгоритм Дейкстры, используя структуру данных \texttt{std::set} для отслеживания активных вершин. Основные шаги:
\begin{itemize}
    \item Инициализация массива \texttt{distances} для хранения минимальных расстояний до вершин.
    \item Отметка всех вершин как непосещённых с помощью массива \texttt{visited}.
    \item Использование множества \texttt{active\_vertices}, в котором хранятся вершины, отсортированные по текущей минимальной стоимости пути.
\item Для каждой вершины:
    \begin{enumerate}
        \item Извлекается вершина с минимальным расстоянием.
        \item Проводится релаксация рёбер, исходящих из текущей вершины.
        \item Если найден новый более короткий путь, расстояние обновляется, а вершина добавляется в множество активных вершин.
    \end{enumerate}
\end{itemize}
После завершения работы алгоритма массив \texttt{res\_} обновляется минимальными расстояниями до всех вершин.

\subsection*{Валидация данных}

Метод \texttt{validation} проверяет:
\begin{itemize}
    \item Наличие входных и выходных данных.
    \item Корректность указателей на массивы данных.
\end{itemize}
Если проверки не пройдены, выполнение алгоритма завершается.

\subsection*{Постобработка данных}

На этапе постобработки методом \texttt{post\_processing} результаты работы алгоритма записываются в выходной массив, чтобы обеспечить передачу данных другим компонентам системы.

\section*{Описание параллельной версии алгоритма}

\subsection*{Конвертация матрицы смежности в формат CRS}

 Основные этапы:
\begin{itemize}
    \item Ненулевые значения рёбер записываются в вектор \texttt{values}.
    \item Индексы целевых вершин для каждого ребра сохраняются в вектор \texttt{columns}.
    \item Границы диапазонов ненулевых элементов для строк матрицы фиксируются в \texttt{row\_ptr}.
\end{itemize}
Дополнительно вычисляются размеры векторов (\texttt{values\_size}, \texttt{columns\_size}, \texttt{row\_ptr\_size}) для дальнейшего распределения данных между процессами.

\subsection*{Предварительная обработка данных}

Метод \texttt{pre\_processing} выполняет следующие действия:
\begin{enumerate}
    \item На процессе с рангом 0 извлекаются матрица смежности и параметры графа (\texttt{V} и \texttt{E}) из входных данных.
    \item Вызывается функция \texttt{CRSconvert} для преобразования данных в формат CRS.
\end{enumerate}

\subsection*{Основной алгоритм}

Метод \texttt{run} реализует параллельную версию алгоритма Дейкстры с использованием библиотеки \texttt{Boost.MPI}. Основные шаги:
\begin{itemize}
    \item Распределение данных:
    \begin{itemize}
        \item Глобальные параметры графа (\texttt{V}, \texttt{E}) и размеры векторов CRS (\texttt{values\_size}, \texttt{columns\_size}, \texttt{row\_ptr\_size}) рассылаются всем процессам.
        \item Вектора \texttt{values}, \texttt{columns}, \texttt{row\_ptr} передаются в процессы с помощью функции \texttt{broadcast}.
    \end{itemize}
    \item Разделение вершин между процессами:
    \begin{itemize}
       \item Каждому процессу назначается диапазон вершин \\
(\texttt{start\_vertex\_index} и \texttt{end\_vertex\_index})
        \item Результаты и метки посещённых вершин инициализируются локально.
    \end{itemize}
    \item Итеративный процесс:
    \begin{enumerate}
        \item Каждый процесс вычисляет локальный минимум расстояния для своего диапазона вершин.
        \item С помощью \texttt{reduce} на корневом процессе определяется глобальный минимум, который рассылается обратно всем процессам (\texttt{broadcast}).
        \item Обновляются расстояния до соседних вершин через релаксацию рёбер.
        \item Процессы синхронизируют результаты через коллективные операции (\texttt{all\_reduce}).
    \end{enumerate}
\end{itemize}

\subsection*{Валидация данных}

Метод \texttt{validation} проверяет:
\begin{itemize}
    \item Наличие входных и выходных данных.
    \item Корректность указателей на массивы данных.
\end{itemize}
\subsection*{Постобработка данных}

На этапе постобработки методом \texttt{post\_processing} результаты работы алгоритма собираются на процессе с рангом 0:
\begin{itemize}
    \item С помощью функции \texttt{all\_reduce} все процессы синхронизируют свои результаты.
    \item Итоговые значения расстояний до вершин записываются в выходной массив, доступный другим компонентам системы.
\end{itemize}
\section*{Сравнение результатов}

В данной секции представлены результаты тестирования алгоритма Дейкстры в последовательной и параллельной версиях, а также проведено сравнение их эффективности при увеличении числа процессов. Тестирование проводилось на следующем аппаратном обеспечении:

\begin{itemize}
    \item Видеокарта: \texttt{Radeon RX 6750 XT}
    \item Процессор: \texttt{AMD Ryzen 5 5600}
    \item Оперативная память: \texttt{32 GB RAM}
    \item Операционная система: \texttt{Windows 11}
\end{itemize}
Для каждой версии алгоритма были собраны данные о времени выполнения, ускорении и эффективности при различных числах процессов. Результаты представлены в виде таблиц , показывающих зависимость этих метрик от числа процессов, а также наглядно демонстрирующие, как параллельная версия алгоритма ускоряет выполнение по сравнению с последовательной.

\subsection*{Тестирование последовательной версии}

Последовательная версия алгоритма Дейкстры была протестирована на одном процессе. Время выполнения алгоритма на различных графах с разными размерами было зафиксировано и использовано для дальнейшего анализа. 

\subsection*{Тестирование параллельной версии}

Параллельная версия алгоритма была протестирована на различных числах процессов. Время выполнения, ускорение и эффективность были рассчитаны для каждого числа четного процесса, начиная с 2-го до 8-го.
\section*{Сравнение результатов}

В таблицах ниже приведены результаты тестирования для различных чисел процессов, включая ускорение и эффективность. Первая таблица отображает результаты для теста \texttt{pipeline}, а вторая — для теста \texttt{task\_run}.

Тесты проводились на графе, содержащем 20000 вершин и 40000 рёбер. Для генерации графа был использован случайный генератор, который заполняет граф следующим образом:

\begin{itemize}
    \item Граф состоит из 20000 вершин и 40000 рёбер, что делает его разрежённым (в среднем каждая вершина соединена с 2 другими вершинами).
    \item Рёбра генерируются случайным образом между вершинами, при этом между одной и той же вершиной самосоединения не допускаются.
    \item Для каждого рёбра случайным образом генерируется вес в пределах от 1 до 100.
    \item Граф представлен в виде матрицы смежности, где для каждой пары вершин, между которыми существует ребро, в соответствующем элементе матрицы хранится вес этого ребра.
\end{itemize}

В процессе тестирования количество активных задач было установлено на уровне 25 (\texttt{perfAttr->num\_running = 25}).

\begin{table}[h!]
\centering
\setlength{\tabcolsep}{3pt} % Уменьшаем ширину столбцов
\small
\begin{tabular}{|c|c|c|c|}  
\hline
\textbf{Число процессов} & \textbf{Время  (с)} & \textbf{Ускорение } & \textbf{Эффективность } \\ \hline
1 & 21.9 & 1.00 & 1.00 \\ \hline
2 & 12.1 & 1.81 & 0.91 \\ \hline
4 & 10.2 & 2.15 & 0.54 \\ \hline
8 & 8.7 & 2.52 & 0.31 \\ \hline
\end{tabular}
\caption{Результаты тестирования для \texttt{pipeline}: время выполнения, ускорение и эффективность}
\end{table}
\vspace*{-1cm} %

\begin{table}[t] % 
\centering
\setlength{\tabcolsep}{3pt} % Уменьшаем ширину столбцов
\small
\begin{tabular}{|c|c|c|c|}  
\hline
\textbf{Число процессов} & \textbf{Время (с)} & \textbf{Ускорение} & \textbf{Эффективность } \\ \hline
1 & 17.05 & 1.00 & 1.00 \\ \hline
2 & 7 & 2.43 & 1.22 \\ \hline
4 & 5.2 & 3.28 & 0.82 \\ \hline
8 & 3.4 & 5.02 & 0.63 \\ \hline
\end{tabular}
\caption{Результаты тестирования для \texttt{task\_run}: время выполнения, ускорение и эффективность}
\end{table}

\clearpage

\section*{Вывод}

На основе проведённого тестирования результаты показали различия в зависимости от числа процессов.

Во-первых, тесты демонстрируют, что с увеличением числа процессов время выполнения снижается. Для алгоритма \texttt{pipeline} можно заметить, что время выполнения значительно уменьшается при переходе от одного процесса (21.9 секунды) к двум (12.1 секунды), но затем снижение замедляется.  Ускорение в тестах увеличивается с ростом числа процессов, однако эффективность начинает падать. Это связано с тем, что при большем числе процессов возникает дополнительная нагрузка на синхронизацию и коммуникацию между процессами.

Для теста \texttt{task\_run} ситуация схожа: время выполнения также сокращается с увеличением числа процессов, начиная с 17.05 секунд для одного процесса и достигая 3.4 секунды для восьми процессов. Ускорение в этом случае увеличивается более существенно, чем для \texttt{pipeline}, достигая 5.02 для восьми процессов. Однако эффективность, как и в предыдущем случае, снижалась с ростом числа процессов.

Таким образом, результаты показывают, что на практике увеличение числа процессов может существенно снизить время выполнения, но при этом эффективность параллельных вычислений уменьшается. Это может быть связано с ограничениями в производительности системы или с накладными расходами на синхронизацию данных между процессами. 

\section{Литература}
\begin{thebibliography}{9}
    \bibitem{dijkstra_geeks} GeeksforGeeks, \textit{Dijkstra’s Shortest Path Algorithm (Greedy Algorithm)}, [Online]. Available: \url{https://www.geeksforgeeks.org/dijkstras-shortest-path-algorithm-greedy-algo-7/}
    \bibitem{dijkstra_favtutor} FavTutor, \textit{Dijkstra’s Algorithm in C++}, [Online]. Available: \url{https://favtutor.com/blogs/dijkstras-algorithm-cpp}
    \bibitem{boost_mpi_collectives} Boost, \textit{Boost.MPI: all\_reduce}, [Online]. Available: \url{https://www.boost.org/doc/libs/1_84_0/boost/mpi/collectives/all_reduce.hpp}
\end{thebibliography}
\section*{Приложение}

\subsection*{Основные функции}

\subsection{Run}
\lstset{language=C++, basicstyle=\footnotesize\ttfamily, frame=single, captionpos=b, numbers=left, numberstyle=\tiny, stepnumber=1, numbersep=5pt, tabsize=2, breaklines=true, breakatwhitespace=false, showspaces=false, showstringspaces=false, morekeywords={size_t}, keywordstyle=\color{blue}\bfseries, stringstyle=\color{green}\ttfamily, commentstyle=\color{red}\ttfamily}

\begin{lstlisting}[caption={Функция run последовательного алгоритма}]
bool kapustin_dijkstras_algorithm_mpi::DijkstrasAlgorithmSEQ::run() {
  internal_order_test();
  std::vector<int> distances(V, INF);
  std::vector<bool> visited(V, false);
  std::set<std::pair<int, int>> active_vertices;

  distances[0] = 0;
  active_vertices.insert({0, 0});

  while (!active_vertices.empty()) {
    int u = active_vertices.begin()->second;
    active_vertices.erase(active_vertices.begin());

    if (visited[u]) continue;
    visited[u] = true;

    for (int j = row_ptr[u]; j < row_ptr[u + 1]; ++j) {
      int v = columns[j];
      int weight = values[j];
      int new_dist = distances[u] + weight;

      if (new_dist < distances[v]) {
        active_vertices.erase({distances[v], v});
        distances[v] = new_dist;
        active_vertices.insert({new_dist, v});
      }
    }
  }

  for (size_t i = 0; i < V; ++i) {
    res_[i] = distances[i];
  }

  return true;
}
\end{lstlisting}
\clearpage
\subsection{Код функций}
\lstset{language=C++, basicstyle=\footnotesize\ttfamily, frame=single, captionpos=b, numbers=left, numberstyle=\tiny, stepnumber=1, numbersep=5pt, tabsize=2, breaklines=true, breakatwhitespace=false, showspaces=false, showstringspaces=false, morekeywords={size_t}, keywordstyle=\color{blue}\bfseries, stringstyle=\color{green}\ttfamily, commentstyle=\color{red}\ttfamily}

\begin{lstlisting}[caption={Функция run параллельного алгоритма}]
bool kapustin_dijkstras_algorithm_mpi::DijkstrasAlgorithmMPI::run() {
  internal_order_test();

  boost::mpi::broadcast(world, V, 0);
  boost::mpi::broadcast(world, E, 0);
  boost::mpi::broadcast(world, values_size, 0);
  boost::mpi::broadcast(world, row_ptr_size, 0);
  boost::mpi::broadcast(world, columns_size, 0);

  values.resize(values_size);
  row_ptr.resize(row_ptr_size);
  columns.resize(columns_size);

  boost::mpi::broadcast(world, values.data(), values.size(), 0);
  boost::mpi::broadcast(world, row_ptr.data(), row_ptr.size(), 0);
  boost::mpi::broadcast(world, columns.data(), columns.size(), 0);

  int vertices_per_process = V / world.size();
  int start_vertex_index = world.rank() * vertices_per_process;
  int end_vertex_index = (world.rank() == world.size() - 1) ? V : start_vertex_index + vertices_per_process;

  res_.resize(V, INF);
  std::vector<bool> visited(V, false);

  if (world.rank() == 0) {
    res_[0] = 0;
  }

  boost::mpi::broadcast(world, res_.data(), V, 0);

  for (size_t iteration = 0; iteration < V; iteration++) {
    int local_min_distance = INF;
    int local_min_vertex = -1;
    int current_vertex = start_vertex_index;

    while (current_vertex < end_vertex_index) {
      if (!visited[current_vertex] && res_[current_vertex] < local_min_distance) {
        local_min_distance = res_[current_vertex];
        local_min_vertex = current_vertex;
      }
      current_vertex++;
    }

    std::pair<int, int> local_min_pair = {local_min_distance, local_min_vertex};
    std::pair<int, int> global_min_pair = {INF, -1};

    boost::mpi::reduce(
        world, local_min_pair, global_min_pair,
        [](const std::pair<int, int>& a, const std::pair<int, int>& b) {
          if (a.first < b.first) return a;
          if (a.first > b.first) return b;
          return (a.second < b.second) ? a : b;
        },
        0);

    boost::mpi::broadcast(world, global_min_pair, 0);

    if (global_min_pair.first == INF) {
      break;
    }

    visited[global_min_pair.second] = true;

    for (int edge_index = row_ptr[global_min_pair.second]; edge_index < row_ptr[global_min_pair.second + 1];
         edge_index++) {
      int neighbor_vertex = columns[edge_index];
      int edge_weight = values[edge_index];

      if (!visited[neighbor_vertex] && res_[global_min_pair.second] != INF &&
          (res_[global_min_pair.second] + edge_weight < res_[neighbor_vertex])) {
        res_[neighbor_vertex] = res_[global_min_pair.second] + edge_weight;
      }
    }
  }
  std::vector<int> global_result(V, INF);
  boost::mpi::all_reduce(world, res_.data(), V, global_result.data(), boost::mpi::minimum<int>());
  res_ = global_result;

  return true;
}
\end{lstlisting}
\clearpage
\subsection{Код функций}
\lstset{language=C++, basicstyle=\footnotesize\ttfamily, frame=single, captionpos=b, numbers=left, numberstyle=\tiny, stepnumber=1, numbersep=5pt, tabsize=2, breaklines=true, breakatwhitespace=false, showspaces=false, showstringspaces=false, morekeywords={size_t}, keywordstyle=\color{blue}\bfseries, stringstyle=\color{green}\ttfamily, commentstyle=\color{red}\ttfamily}

\begin{lstlisting}[caption={Функция CRSconvert для преобразования матрицы в формат CRS. MPI версия.}]
void kapustin_dijkstras_algorithm_mpi::DijkstrasAlgorithmMPI::CRSconvert(const int* input_matrix) {
  row_ptr.resize(V + 1);
  size_t non_zero_count = 0;

  for (size_t i = 0; i < V; ++i) {
    const int* row_start = input_matrix + i * V;
    const int* row_end = row_start + V;

    for (; row_start != row_end; ++row_start) {
      if (*row_start != 0) {
        values.push_back(*row_start);
        columns.push_back(row_start - (input_matrix + i * V));
        ++non_zero_count;
      }
    }
    row_ptr[i + 1] = non_zero_count;
  }

  values_size = values.size();
  columns_size = columns.size();
  row_ptr_size = row_ptr.size();
}
\end{lstlisting}
\clearpage
\subsection{Пример теста для алгоритма Дейкстры}

\lstset{language=C++, basicstyle=\footnotesize\ttfamily, frame=single, captionpos=b, numbers=left, numberstyle=\tiny, stepnumber=1, numbersep=5pt, tabsize=2, breaklines=true, breakatwhitespace=false, showspaces=false, showstringspaces=false, morekeywords={size_t}, keywordstyle=\color{blue}\bfseries, stringstyle=\color{green}\ttfamily, commentstyle=\color{red}\ttfamily}

\begin{lstlisting}[caption={Пример теста  на разорванном графе}]
TEST(kapustin_dijkstras_algorithm_mpi, test_disconnected_graph) {
  boost::mpi::communicator world;
  const int V = 4;
  const int E = 2;

  std::vector<int> graph = {0, 1, 0, 0, 1, 0, 0, 0, 0, 0, 0, 2, 0, 0, 2, 0};
  std::vector<int> resMPI(V, 0);
  std::vector<int> expected_res = {0, 1, INT_MAX, INT_MAX};

  std::shared_ptr<ppc::core::TaskData> taskDataPar = std::make_shared<ppc::core::TaskData>();

  taskDataPar->inputs.emplace_back(reinterpret_cast<uint8_t *>(graph.data()));
  taskDataPar->inputs_count.emplace_back(static_cast<uint32_t>(V));
  taskDataPar->inputs_count.emplace_back(static_cast<uint32_t>(E));
  taskDataPar->outputs.emplace_back(reinterpret_cast<uint8_t *>(resMPI.data()));
  taskDataPar->outputs_count.emplace_back(resMPI.size());

  kapustin_dijkstras_algorithm_mpi::DijkstrasAlgorithmMPI testMpiTaskParallel(taskDataPar);

  ASSERT_EQ(testMpiTaskParallel.validation(), true);

  testMpiTaskParallel.pre_processing();
  testMpiTaskParallel.run();
  testMpiTaskParallel.post_processing();

  if (world.rank() == 0) {
    ASSERT_EQ(resMPI, expected_res);
  }
}
\end{lstlisting}
\clearpage
\subsection{Pipeline тест}

\lstset{language=C++, basicstyle=\footnotesize\ttfamily, frame=single, captionpos=b, numbers=left, numberstyle=\tiny, stepnumber=1, numbersep=5pt, tabsize=2, breaklines=true, breakatwhitespace=false, showspaces=false, showstringspaces=false, morekeywords={size_t}, keywordstyle=\color{blue}\bfseries, stringstyle=\color{green}\ttfamily, commentstyle=\color{red}\ttfamily}

\begin{lstlisting}[caption={Пример теста для pipeline}]
TEST(kapustin_dijkstras_algorithm_mpi, test_pipeline_run) {
  const int V = 20000;
  const int E = 40000;
  boost::mpi::communicator world;
  std::vector<int> graph = generateGraph::generateGraph(V, E);
  std::vector<int> resMPI(V, 0);
  std::shared_ptr<ppc::core::TaskData> taskDataMPI = std::make_shared<ppc::core::TaskData>();
  taskDataMPI->inputs.emplace_back(reinterpret_cast<uint8_t*>(graph.data()));
  taskDataMPI->inputs_count.emplace_back(static_cast<uint32_t>(V));
  taskDataMPI->inputs_count.emplace_back(static_cast<uint32_t>(E));
  taskDataMPI->outputs.emplace_back(reinterpret_cast<uint8_t*>(resMPI.data()));
  taskDataMPI->outputs_count.emplace_back(resMPI.size());
  auto testTaskMPI = std::make_shared<kapustin_dijkstras_algorithm_mpi::DijkstrasAlgorithmMPI>(taskDataMPI);
  auto perfAttr = std::make_shared<ppc::core::PerfAttr>();
  perfAttr->num_running = 25;
  const auto t0 = std::chrono::high_resolution_clock::now();
  perfAttr->current_timer = [&] {
    auto current_time_point = std::chrono::high_resolution_clock::now();
    auto duration = std::chrono::duration_cast<std::chrono::nanoseconds>(current_time_point - t0).count();
    return static_cast<double>(duration) * 1e-9;
  };

  auto perfResults = std::make_shared<ppc::core::PerfResults>();

  ppc::core::Perf perfAnalyzer(testTaskMPI);
  perfAnalyzer.pipeline_run(perfAttr, perfResults);

  if (world.rank() == 0) {
    ppc::core::Perf::print_perf_statistic(perfResults);
  }
}
\end{lstlisting}
\clearpage
\subsection{Taskrun тест}

\lstset{language=C++, basicstyle=\footnotesize\ttfamily, frame=single, captionpos=b, numbers=left, numberstyle=\tiny, stepnumber=1, numbersep=5pt, tabsize=2, breaklines=true, breakatwhitespace=false, showspaces=false, showstringspaces=false, morekeywords={size_t}, keywordstyle=\color{blue}\bfseries, stringstyle=\color{green}\ttfamily, commentstyle=\color{red}\ttfamily}

\begin{lstlisting}[caption={Пример теста для taskrun}]
TEST(kapustin_dijkstras_algorithm_mpi, test_task_run) {
  const int V = 20000;
  const int E = 40000;
  boost::mpi::communicator world;
  std::vector<int> graph = generateGraph::generateGraph(V, E);
  std::vector<int> resMPI(V, 0);

  std::shared_ptr<ppc::core::TaskData> taskDataMPI = std::make_shared<ppc::core::TaskData>();
  taskDataMPI->inputs.emplace_back(reinterpret_cast<uint8_t*>(graph.data()));
  taskDataMPI->inputs_count.emplace_back(static_cast<uint32_t>(V));
  taskDataMPI->inputs_count.emplace_back(static_cast<uint32_t>(E));
  taskDataMPI->outputs.emplace_back(reinterpret_cast<uint8_t*>(resMPI.data()));
  taskDataMPI->outputs_count.emplace_back(resMPI.size());
  auto testTaskMPI = std::make_shared<kapustin_dijkstras_algorithm_mpi::DijkstrasAlgorithmMPI>(taskDataMPI);
  auto perfAttr = std::make_shared<ppc::core::PerfAttr>();
  perfAttr->num_running = 25;
  const auto t0 = std::chrono::high_resolution_clock::now();
  perfAttr->current_timer = [&] {
    auto current_time_point = std::chrono::high_resolution_clock::now();
    auto duration = std::chrono::duration_cast<std::chrono::nanoseconds>(current_time_point - t0).count();
    return static_cast<double>(duration) * 1e-9;
  };

  auto perfResults = std::make_shared<ppc::core::PerfResults>();

  ppc::core::Perf perfAnalyzer(testTaskMPI);
  perfAnalyzer.task_run(perfAttr, perfResults);

  if (world.rank() == 0) {
    ppc::core::Perf::print_perf_statistic(perfResults);
  }
}
\end{lstlisting}

\end{document}